\documentclass[aspectratio=169]{beamer}
\usetheme{metropolis}
\metroset{subsectionpage=progressbar}

\usefonttheme{professionalfonts} % 使用系统/自定字体


% === 字体设置 ===
\usepackage[UTF8,scheme=plain,fontset=none]{ctex}
\setCJKmainfont{Source Han Serif CN}[BoldFont={Source Han Serif CN Bold}]
\setCJKsansfont{Source Han Sans CN}[BoldFont={Source Han Sans CN Bold}]
% \setCJKmonofont{Sarasa Mono CN}

% \input{commands.tex}

% beamer 已加载 hyperref;加 unicode 以支持中文书签
\hypersetup{unicode}

% define paragraph
\providecommand{\paragraph}[1]{\smallskip\textbf{#1}\par}

% 常用包
\usepackage{longtable,booktabs}
\usepackage{amsmath,amssymb}
\usepackage{graphicx}
\usepackage{tikz}
\usetikzlibrary{positioning,arrows.meta}
% \graphicspath{{.}{./figs/}{./images/}{./images_in_paper/}}
\usepackage{caption}
\usepackage{subcaption}
\usepackage{float}
\usepackage{svg}
\usepackage{booktabs}
\usepackage{array}
\usepackage{threeparttable}

% 算法环境(与 beamer 兼容)
\usepackage{algorithm}
\usepackage[noend]{algpseudocode}  % 提供 algorithmic 环境、\State 等
% 可选:微调 algorithmic 缩进
\algrenewcommand\algorithmicindent{0.8em}

% 数学粗体与梯度符号
\usepackage{bm} % \boldsymbol
\newcommand{\bx}{\mathbf{x}}
\newcommand{\bz}{\mathbf{z}}
\newcommand{\bI}{\mathbf{I}}
\newcommand{\bzero}{\mathbf{0}}
\newcommand{\bepsilon}{\boldsymbol{\epsilon}}
\newcommand{\grad}{\nabla}
% 超链接(beamer 已加载 hyperref,这里只补选项)
% \hypersetup{unicode=true}

% 编号风格
\setbeamertemplate{caption}[numbered]
\setbeamertemplate{caption label separator}{.}

\title{}
\author{分类问题视角下的AdaBoost算法}
\date{\today}

%---Document Begins---
\begin{document}
\begin{frame}[plain]
  \titlepage
\end{frame}

\section{AdaBoost模型结构}

\begin{frame}{模型介绍}
  % 思想性地描述 AdaBoost 的整体结构
  \begin{itemize}
    \item AdaBoost 是一种典型的提升(Boosting)型集成学习框架,
      其核心思想是:\textbf{通过多轮训练、聚焦难样本,把一群“弱学习器”提升为一个“强学习器”。}
    \item 在每一轮中,AdaBoost 都会为总模型增加一个新的学习器,指导模型的弱学习器个数达到预先指定的值。
    \item 训练新学习器时,根据上一轮的推理结果在同一训练集上\textbf{重新分配样本权重},
      使新的弱学习器更加关注上一轮中被分错或“难学”的样本。
    \item 各轮得到的弱学习器本身能力都比较弱,但在最后通过加权组合(加权投票或加权求和),
      形成一个整体性能更高、泛化能力更强的强学习器。
    \item AdaBoost 不限定弱学习器的具体形式(如决策树桩、小深度决策树等),因此可以看作一个\textbf{通用的、
      可移植的集成学习框架}。
  \end{itemize}
\end{frame}

\begin{frame}{AdaBoost模型结构图}
  \begin{figure}[H]
    \centering
    \includegraphics[width=0.9\linewidth]{./assets/adaboost.png}
    \caption{AdaBoost模型结构图}
  \end{figure}
\end{frame}

\subsection{计算流程}

\begin{frame}{错误率计算规则}
  假设训练集包含 $m$ 个样本 $\{(x_i,y_i),\cdots,(x_m,y_m)\}$。

  不失一般性地,假设现在训练已经进行到第 $t$ 轮,将要训练第 $t$ 个弱学习器。

  记第 $t$ 个学习器在样本 $x$ 上的预测结果为 $h_t(x)$。

  设训练第 $t$ 个弱学习器所使用的样本权重为
  \[
    D_t = \{\omega_{t1},\cdots,\omega_{tm}\},\qquad
    \sum_{i=1}^m \omega_{ti} = 1.
  \]

  则第 $t$ 轮弱学习器的加权错误率为
  \[
    \varepsilon_t
      = \sum_{i=1}^m \omega_{ti}\,\mathbf{1}\{h_t(x_i)\neq y_i\}.
  \]
\end{frame}


\begin{frame}{样本加权规则}
  \textbf{第 $t$ 轮($t=1,\dots,T$):}
  \begin{enumerate}
    \item 计算系数
      \[
        \beta_t = \frac{\varepsilon_t}{1-\varepsilon_t},
        \qquad 0 < \beta_t < 1.
      \]
    \item 先得到更新后的未归一化权重
      \[
        \tilde{\omega}_{t+1,i}
          = \omega_{ti}\,\beta_t^{\,1-\mathbf{1}\{h_t(x_i)\neq y_i\}},
          \qquad i=1,\dots,m.
      \]
    \item 再将其归一化,得到下一轮的权重分布
      \[
        \omega_{t+1,i}
          = \frac{\tilde{\omega}_{t+1,i}}
                  {\sum_{j=1}^m \tilde{\omega}_{t+1,j}}.
      \]
  \end{enumerate}

  \vspace{0.5em}
  \textbf{直观理解:}
  \[
    \begin{cases}
      h_t(x_i)=y_i
      \Rightarrow \tilde{\omega}_{t+1,i}=\omega_{ti}\,\beta_t
        & \text{(分对:权重减小)}\\[0.3em]
      h_t(x_i)\neq y_i
      \Rightarrow \tilde{\omega}_{t+1,i}=\omega_{ti}
        & \text{(分错:权重不变,归一化后相对增大)}
    \end{cases}
  \]
\end{frame}

\section{收敛性}

\begin{frame}{AdaBoost 的训练误差界}

\textbf{设:}
\begin{itemize}
  \item 第 $t$ 轮弱学习器的加权错误率为 $\varepsilon_t$;
  \item 最终强分类器为 $h_f$,其在训练分布 $D$ 下的错误率为
    \[
      \varepsilon = \Pr_{i\sim D}[h_f(x_i)\neq y_i].
    \]
\end{itemize}

\end{frame}
\begin{frame}{AdaBoost 的训练误差界}
\textbf{训练误差上界}
\[
  \varepsilon \;\le\; 2^T \prod_{t=1}^T \sqrt{\varepsilon_t(1-\varepsilon_t)}.
\]

引入第 $t$ 轮的“优势”(edge):比随机猜测强的部分
\[
  \gamma_t = \tfrac{1}{2}-\varepsilon_t,
\]
则上界可以改写为
\[
  \varepsilon
  \;\le\; \prod_{t=1}^T \sqrt{1-4\gamma_t^2}
  = \exp\!\Bigl(-\sum_{t=1}^T \mathrm{KL}\!\bigl(\tfrac12 \,\big\|\, \tfrac12-\gamma_t\bigr)\Bigr)
  \;\le\; \exp\!\Bigl(-2\sum_{t=1}^T \gamma_t^2\Bigr).
\]

\end{frame}

\begin{frame}{AdaBoost 的训练误差界}
\textbf{特殊情形:} 若所有弱学习器的错误率都相同,
$\varepsilon_t = \tfrac12-\gamma$($\gamma>0$),则
\[
  \varepsilon \;\le\; (1-4\gamma^2)^{T/2}
  = \exp\!\bigl(-T\cdot \mathrm{KL}(\tfrac12 \,\|\, \tfrac12-\gamma)\bigr)
  \;\le\; \exp(-2T\gamma^2).
\]

\textbf{结论:} 只要每一轮的弱学习器都略好于随机猜测($\gamma_t>0$),
AdaBoost 在训练集上的错误率会随轮数 $T$ \alert{指数级下降}。

\end{frame}

\subsection{收敛性证明}

\begin{frame}{收敛性证明}
  \textbf{引理:} 训练误差界受限于归一化因子之积。
  \[
    \frac{1}{m}\sum_{i=1}^m \exp(-y_i f(x_i)) = \prod_{t=1}^T Z_t.
  \]
  其中 $Z_t = \sum_{i=1}^m \omega_{ti} \exp(-\alpha_t y_i h_t(x_i))$。
  
  \textbf{证明思路:}
  \begin{enumerate}
    \item 递推关系:$\omega_{t+1,i} = \frac{\omega_{ti} \exp(-\alpha_t y_i h_t(x_i))}{Z_t}$
    \item 展开得到:$\omega_{T+1,i} = \frac{1}{m \prod_{t=1}^T Z_t} \exp(-y_i \sum_{t=1}^T \alpha_t h_t(x_i))$
    \item 由 $\sum \omega_{T+1,i} = 1$,可得上述等式。
    \item 由于 $\mathbf{1}\{H(x)\neq y\} \le \exp(-y H(x))$,故误差 $\le \prod Z_t$。
  \end{enumerate}
\end{frame}

\begin{frame}{收敛性证明(续)}
  \textbf{计算 $Z_t$ 的极小值:}
  \[
    Z_t = \sum_{y_i=h_t(x_i)} \omega_{ti} e^{-\alpha_t} + \sum_{y_i \neq h_t(x_i)} \omega_{ti} e^{\alpha_t}
    = (1-\varepsilon_t)e^{-\alpha_t} + \varepsilon_t e^{\alpha_t}
  \]
  对 $\alpha_t$ 求导并令其为0,可得最优权重:
  \[
    \alpha_t = \frac{1}{2} \ln \frac{1-\varepsilon_t}{\varepsilon_t}
  \]
  代回 $Z_t$ 表达式:
  \[
    Z_t = 2\sqrt{\varepsilon_t(1-\varepsilon_t)} = \sqrt{1-4\gamma_t^2}
  \]
  从而得证:
  \[
    \varepsilon_{\text{train}} \le \prod_{t=1}^T \sqrt{1-4\gamma_t^2} \le \exp(-2\sum_{t=1}^T \gamma_t^2)
  \]
\end{frame}

\section{泛化误差}

\begin{frame}{泛化误差与过拟合}
  \begin{itemize}
    \item \textbf{现象:} 在许多实验中观察到,即使训练误差已经降为0,继续增加弱分类器数量 $T$,测试误差不仅没有上升(过拟合),反而进一步下降。
    \item \textbf{传统解释失效:} 传统的 VC 维理论认为模型复杂度随 $T$ 增加,泛化界应变差。
    \item \textbf{间隔理论(Margin Theory):} 
      AdaBoost 在训练误差为 0 后,继续训练会使得样本的\textbf{分类间隔(Margin)}不断增大。
      \[
        \text{margin}(x, y) = \frac{y \sum_t \alpha_t h_t(x)}{\sum_t \alpha_t}
      \]
    \item 更大的间隔意味着更强的鲁棒性和更好的泛化能力。
  \end{itemize}
\end{frame}

\section{理论进阶}

\begin{frame}{统计解释:指数损失最小化}
  AdaBoost 的另一种解释是:\textbf{前向分步加法模型}(Forward Stagewise Additive Modeling)在\textbf{指数损失函数}下的优化过程。
  
  \begin{itemize}
    \item \textbf{加法模型:} $f(x) = \sum_{t=1}^T \alpha_t h_t(x)$
    \item \textbf{损失函数:} $L(y, f(x)) = \exp(-y f(x))$
  \end{itemize}

  \textbf{性质:}
  指数损失是 0/1 损失的一致上界,且处处可微。
  \[
    \mathbf{1}\{y \neq f(x)\} \le \exp(-y f(x))
  \]
  最小化指数损失等价于最小化分类错误率的上界。这为 AdaBoost 的权重更新规则提供了统计学依据。
\end{frame}

\begin{frame}{多分类扩展:SAMME 算法}
  原始 AdaBoost (AdaBoost.M1) 主要针对二分类问题。对于 MNIST 手写数字识别(10分类),需要进行扩展。
  
  \textbf{SAMME (Stagewise Additive Modeling using a Multi-class Exponential loss function)} 算法调整了权重更新公式:
  \[
    \alpha_t = \ln \frac{1-\varepsilon_t}{\varepsilon_t} + \ln(K-1)
  \]
  其中 $K$ 是类别数量(MNIST 中 $K=10$)。
  
  \textbf{收敛条件:}
  只要 $\varepsilon_t < 1 - \frac{1}{K}$(即弱分类器优于随机猜测),则 $\alpha_t > 0$,算法即可收敛。
\end{frame}

\begin{frame}{偏差-方差权衡 (Bias-Variance Tradeoff)}
  从误差分解的视角对比 Bagging (如随机森林) 与 Boosting (如 AdaBoost):
  \begin{itemize}
    \item \textbf{Bagging (并行):} 
      通过对训练集重采样训练多个强学习器并取平均。
      主要降低\textbf{方差 (Variance)},适合高方差模型(如完全生长的决策树)。
    \item \textbf{Boosting (串行):} 
      通过迭代修正前一轮的错误。
      主要降低\textbf{偏差 (Bias)},能将高偏差的\textbf{弱学习器}(如决策树桩)提升为强学习器。
      随着迭代次数增加,方差也可能略有下降,但过度迭代会导致过拟合(方差上升)。
  \end{itemize}
\end{frame}

\begin{frame}{正则化:Shrinkage 策略}
  为了进一步防止过拟合,AdaBoost 常引入\textbf{学习率 (Learning Rate)} 参数 $\nu$ ($0 < \nu \le 1$):
  \[
    f_t(x) = f_{t-1}(x) + \nu \cdot \alpha_t h_t(x)
  \]
  \begin{itemize}
    \item \textbf{作用:} 限制每个弱分类器的贡献,迫使学习过程更加缓慢和稳健。
    \item \textbf{代价:} 较小的 $\nu$ 通常意味着需要更多的迭代次数 $T$ 才能达到同样的训练误差。
    \item 在实践中,这是调节模型泛化能力最重要的参数之一。
  \end{itemize}
\end{frame}

\section{总结与展望}

\begin{frame}{AdaBoost 优缺点总结}
  \begin{columns}[T]
    \column{0.48\textwidth}
      \textbf{优点:}
      \begin{itemize}
        \item \textbf{泛化能力强:} 在许多问题上不易过拟合(Margin 理论)。
        \item \textbf{参数少:} 原始算法几乎无需调参。
        \item \textbf{通用性:} 可与任何弱学习器结合。
      \end{itemize}
    \column{0.48\textwidth}
      \textbf{缺点:}
      \begin{itemize}
        \item \textbf{对噪声敏感:} 异常值权重会被过度放大(本次实验重点验证)。
        \item \textbf{串行训练:} 难以并行化,训练速度较慢。
      \end{itemize}
  \end{columns}
  
  \vspace{1em}
  \textbf{经典应用:} Viola-Jones 人脸检测框架(基于 Haar 特征 + AdaBoost 级联)。
\end{frame}

\section{任务:手写数字识别}
\begin{frame}{训练流程及结果}
  数据集

  关键代码
  
  准确率
\end{frame}
\subsection{误差分析}
\begin{frame}{基于MNIST变体的鲁棒性分析}
  为了评估模型的鲁棒性,我们将原始 MNIST 测试集进行不同程度的变换作为新的测试集(MNIST Variants),主要包括噪声干扰(高斯噪声、椒盐噪声)和几何变换(旋转、缩放)。
  
  \textbf{实验结果与分析:}
  \begin{itemize}
    \item \textbf{对噪声敏感:} 
      AdaBoost 在干净数据上表现优异,但随着噪声比例增加,准确率下降明显快于 Bagging 类算法(如随机森林)。
    \item \textbf{原因探究:} 
      AdaBoost 的核心机制是关注“难分样本”。高噪声样本往往被视为难分样本,权重被不断放大。
    \item \textbf{权重偏移:}
      模型过度关注这些无法正确分类的噪声点(离群点),导致决策边界发生扭曲,从而降低了对正常样本的泛化能力。
  \end{itemize}
\end{frame}
\begin{frame}[standout]
  谢谢大家!
\end{frame}
\end{document}
