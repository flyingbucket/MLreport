\documentclass[aspectratio=169]{beamer}
\usetheme{metropolis}
\metroset{subsectionpage=progressbar}

\usefonttheme{professionalfonts} % 使用系统/自定字体


% === 字体设置 ===
\usepackage[UTF8,scheme=plain,fontset=none]{ctex}
\setCJKmainfont{Source Han Serif CN}[BoldFont={Source Han Serif CN Bold}]
\setCJKsansfont{Source Han Sans CN}[BoldFont={Source Han Sans CN Bold}]
% \setCJKmonofont{Sarasa Mono CN}

% \input{commands.tex}

% beamer 已加载 hyperref;加 unicode 以支持中文书签
\hypersetup{unicode}

% define paragraph
\providecommand{\paragraph}[1]{\smallskip\textbf{#1}\par}

% 常用包
\usepackage{longtable,booktabs}
\usepackage{amsmath,amssymb}
\usepackage{graphicx}
\usepackage{tikz}
\usetikzlibrary{positioning,arrows.meta}
% \graphicspath{{.}{./figs/}{./images/}{./images_in_paper/}}
\usepackage{caption}
\usepackage{subcaption}
\usepackage{float}
\usepackage{svg}
\usepackage{booktabs}
\usepackage{array}
\usepackage{threeparttable}

% 算法环境(与 beamer 兼容)
\usepackage{algorithm}
\usepackage[noend]{algpseudocode}  % 提供 algorithmic 环境、\State 等
% 可选:微调 algorithmic 缩进
\algrenewcommand\algorithmicindent{0.8em}

% 数学粗体与梯度符号
\usepackage{bm} % \boldsymbol
\newcommand{\bx}{\mathbf{x}}
\newcommand{\bz}{\mathbf{z}}
\newcommand{\bI}{\mathbf{I}}
\newcommand{\bzero}{\mathbf{0}}
\newcommand{\bepsilon}{\boldsymbol{\epsilon}}
\newcommand{\grad}{\nabla}
% 超链接(beamer 已加载 hyperref,这里只补选项)
% \hypersetup{unicode=true}

% 编号风格
\setbeamertemplate{caption}[numbered]
\setbeamertemplate{caption label separator}{.}

\title{}
\author{分类问题视角下的AdaBoost算法}
\date{\today}

%---Document Begins---
\begin{document}
\begin{frame}[plain]
  \titlepage
\end{frame}

\section{AdaBoost模型结构}

\begin{frame}{模型介绍}
  % 思想性地描述 AdaBoost 的整体结构
  \begin{itemize}
    \item AdaBoost 是一种典型的提升(Boosting)型集成学习框架,
      其核心思想是:\textbf{通过多轮训练、聚焦难样本,把一群“弱学习器”提升为一个“强学习器”。}
    \item 在每一轮中,AdaBoost 都会为总模型增加一个新的学习器,指导模型的弱学习器个数达到预先指定的值。
    \item 训练新学习器时,根据上一轮的推理结果在同一训练集上\textbf{重新分配样本权重},
      使新的弱学习器更加关注上一轮中被分错或“难学”的样本。
    \item 各轮得到的弱学习器本身能力都比较弱,但在最后通过加权组合(加权投票或加权求和),
      形成一个整体性能更高、泛化能力更强的强学习器。
    \item AdaBoost 不限定弱学习器的具体形式(如决策树桩、小深度决策树等),因此可以看作一个\textbf{通用的、
      可移植的集成学习框架}。
  \end{itemize}
\end{frame}

\begin{frame}{AdaBoost模型结构图}
  \begin{figure}[H]
    \centering
    \includegraphics[width=0.9\linewidth]{./assets/adaboost.png}
    \caption{AdaBoost模型结构图}
  \end{figure}
\end{frame}

\subsection{计算流程}

\begin{frame}{错误率计算规则}
  假设训练集包含 $m$ 个样本 $\{(x_i,y_i),\cdots,(x_m,y_m)\}$。

  不失一般性地,假设现在训练已经进行到第 $t$ 轮,将要训练第 $t$ 个弱学习器。

  记第 $t$ 个学习器在样本 $x$ 上的预测结果为 $h_t(x)$。

  设训练第 $t$ 个弱学习器所使用的样本权重为
  \[
    D_t = \{\omega_{t1},\cdots,\omega_{tm}\},\qquad
    \sum_{i=1}^m \omega_{ti} = 1.
  \]

  则第 $t$ 轮弱学习器的加权错误率为
  \[
    \varepsilon_t
      = \sum_{i=1}^m \omega_{ti}\,\mathbf{1}\{h_t(x_i)\neq y_i\}.
  \]
\end{frame}


\begin{frame}{样本加权规则}
  \textbf{第 $t$ 轮($t=1,\dots,T$):}
  \begin{enumerate}
    \item 计算系数
      \[
        \beta_t = \frac{\varepsilon_t}{1-\varepsilon_t},
        \qquad 0 < \beta_t < 1.
      \]
    \item 先得到更新后的未归一化权重
      \[
        \tilde{\omega}_{t+1,i}
          = \omega_{ti}\,\beta_t^{\,1-\mathbf{1}\{h_t(x_i)\neq y_i\}},
          \qquad i=1,\dots,m.
      \]
    \item 再将其归一化,得到下一轮的权重分布
      \[
        \omega_{t+1,i}
          = \frac{\tilde{\omega}_{t+1,i}}
                  {\sum_{j=1}^m \tilde{\omega}_{t+1,j}}.
      \]
  \end{enumerate}

  \vspace{0.5em}
  \textbf{直观理解:}
  \[
    \begin{cases}
      h_t(x_i)=y_i
      \Rightarrow \tilde{\omega}_{t+1,i}=\omega_{ti}\,\beta_t
        & \text{(分对:权重减小)}\\[0.3em]
      h_t(x_i)\neq y_i
      \Rightarrow \tilde{\omega}_{t+1,i}=\omega_{ti}
        & \text{(分错:权重不变,归一化后相对增大)}
    \end{cases}
  \]
\end{frame}

\section{收敛性}

\begin{frame}{AdaBoost 的训练误差界}

\textbf{设:}
\begin{itemize}
  \item 第 $t$ 轮弱学习器的加权错误率为 $\varepsilon_t$;
  \item 最终强分类器为 $h_f$,其在训练分布 $D$ 下的错误率为
    \[
      \varepsilon = \Pr_{i\sim D}[h_f(x_i)\neq y_i].
    \]
\end{itemize}

\end{frame}
\begin{frame}{AdaBoost 的训练误差界}
\textbf{训练误差上界}
\[
  \varepsilon \;\le\; 2^T \prod_{t=1}^T \sqrt{\varepsilon_t(1-\varepsilon_t)}.
\]

引入第 $t$ 轮的“优势”(edge):比随机猜测强的部分
\[
  \gamma_t = \tfrac{1}{2}-\varepsilon_t,
\]
则上界可以改写为
\[
  \varepsilon
  \;\le\; \prod_{t=1}^T \sqrt{1-4\gamma_t^2}
  = \exp\!\Bigl(-\sum_{t=1}^T \mathrm{KL}\!\bigl(\tfrac12 \,\big\|\, \tfrac12-\gamma_t\bigr)\Bigr)
  \;\le\; \exp\!\Bigl(-2\sum_{t=1}^T \gamma_t^2\Bigr).
\]

\end{frame}

\begin{frame}{AdaBoost 的训练误差界}
\textbf{特殊情形:} 若所有弱学习器的错误率都相同,
$\varepsilon_t = \tfrac12-\gamma$($\gamma>0$),则
\[
  \varepsilon \;\le\; (1-4\gamma^2)^{T/2}
  = \exp\!\bigl(-T\cdot \mathrm{KL}(\tfrac12 \,\|\, \tfrac12-\gamma)\bigr)
  \;\le\; \exp(-2T\gamma^2).
\]

\textbf{结论:} 只要每一轮的弱学习器都略好于随机猜测($\gamma_t>0$),
AdaBoost 在训练集上的错误率会随轮数 $T$ \alert{指数级下降}。

\end{frame}

\subsection{收敛速度证明(可选)}

\section{任务:手写数字识别}
\begin{frame}{训练流程及结果}
  数据集

  关键代码
  
  准确率
\end{frame}
\subsection{误差分析}
\begin{frame}
  针对AdaBoost对噪声(离群点)敏感的问题,从过拟合的角度入手分析

  AdaBoost会逐渐给离群样本赋予更高的权重,追踪样本权重的变化

  MINIST数据集貌似是清洗过的,我们可以自行添加噪声来呈现AdaBoost的过拟合效应
\end{frame}
\begin{frame}[standout]
  谢谢大家!
\end{frame}
\end{document}
